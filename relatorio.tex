\documentclass{article}

%\usepackage[margin=3.7cm]{geometry}
\usepackage[utf8]{inputenc} 
\usepackage[T1]{fontenc}
\usepackage[brazil]{babel}
\usepackage{mathtools, amssymb} %{amsmath}
\usepackage{float}
\usepackage{graphicx}
\usepackage{wrapfig}
\usepackage[style=brazilian]{csquotes}
\usepackage{xcolor}


% TODO: Apagar
\usepackage[colorinlistoftodos, color=yellow]{todonotes}

\graphicspath{ {Imagens/} }


\begin{document}
	\begin{wrapfigure}{L}{0.3\textwidth}
		\begin{flushleft}	
			\includegraphics[height=.065\textheight]{PESC.png}
		\end{flushleft}
	\end{wrapfigure}
	
	\quad\\
	{Universidade Federal do Rio de Janeiro} \\
	{Otimização Combinatória - COS890} \\
	{Professor Abílio Lucena}
	
	\quad\\
	\vspace*{2cm}
	
	\begin{center}
		\huge\bfseries
		Empacotamento de Conjuntos
		\todo{Verificar o título...}
	\end{center}
	\vspace*{3mm}
	
	\begin{center}
		\large
		Amanda Camacho Novaes de Oliveira
		
		Diego Amaro	
		
		Diego Athayde \todo{Completar os nomes}
	\end{center}

	\vspace{1cm}
	
	\section{Introdução}
	\todo{Título pode ser \enquote{O problema, ou algo assim?}}
	Descrever o problema (dar o exemplo de leiloes combinatorios?)
	
	\todo[inline]{Formular aqui de uma vez ou criar seção só pra isso? Descrição do contexto e modelagem do mesmo não são necessariamente o mesmo...}
	
	
	
	\section{Relaxação Lagrangeana}
	
	\section{Branch and Bound}
	
	\section{Branch and Cut}
	
	\section{Resultados}
	
	\section{Discussão}
	
	\section{Conclusão}

	
\end{document}